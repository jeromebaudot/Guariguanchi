\section{Introduction}
\label{sec:Intro}

{\guari} is a package created in the context of the ILD collaboration~\cite{bib:ILDcoll} and managed by the PICSEL group at IPHC (Strasbourg).
It actually started as a tool for calculating telescope's pointing resolution at device under test (DUT) position for beam-test setups. It then quickly 
transformed for the more challenging task of tracker optimization by comparing tracking performances of different detector configurations.

The package has a set of tools which allows to calculate for a given momentum $(p,\theta,\phi)$ the track parameters covariance matrix. It also allows to 
calculate a tracking efficiency with simplified pattern recognition algorithms. The details of the calculations are explained in section~\ref{sec:Math_formalism}.
Of course, these calculations are performed with simplifying assumptions and ignore other effects that can only be taken into account with full-simulation 
and realistic pattern recognition algorithms. Detector optimization can be guided by plotting the tracking performances as a function momentum, polar and 
azimuthal angles for the different detector configurations.

\subsection{General philosophy}

{\guari} is a C++ software meant as a lightweight package in which only {\tt ROOT}~\cite{bib:root} pre-installation is required. During installation an executable 
is produced which is later run with a configuration data-card. This data-card contains all the configuration variables for the analyses to be performed on a list of 
geometries. The work-flow is as follows,

\begin{itemize}

 \item  Readout of the analysis configuration variables, {\it e.g.} particle type and origin, momentum ranges in terms of $(p,\theta,\phi)$, 
 analysis variables, and list of geometry data-cards (geo-datacard).
 
 \item  Geometries build-up. The geo-data-cards are readout to build a list of geometries.
 
 \item  Each geometry has a so-called tracker associated to it. This object is the one able to perform all the tracking performances 
 calculations related with the geometry, {\it e.g.} track intersections with geometry (including sensitive and insensitive elements), space points 
 covariance matrix (including multiple scattering and energy loss fluctuations), track parameters covariance matrix and tracking efficiency.
 
 \item  Depending of the analysis configuration, different plots can be produced: geometry visualization (including some tracks if specified), 
 track parameters resolution  and/or tracking efficiency vs $(p,\theta,\phi)$.
 
\end{itemize}

The resulting plots are always stored in a multiple pages pdf file, and can if requested be stored in an output root file as well.